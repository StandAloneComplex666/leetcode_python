% !TEX program = xelatex

\documentclass{resume}
%\usepackage{zh_CN-Adobefonts_external} % Simplified Chinese Support using external fonts (./fonts/zh_CN-Adobe/)
%\usepackage{zh_CN-Adobefonts_internal} % Simplified Chinese Support using system fonts

\begin{document}
\pagenumbering{gobble} % suppress displaying page number

\name{Yancheng Chen}

\section{Basic Information}

\begin{tabular}{rl}
    \textsc{Address:}   & 724 5th Ave, Brooklyn, New York,NY 11232 \\
    \textsc{Phone:}     & 917-856-9718\\
    \textsc{email:}     & yc2739@nyu.edu \href{mailto:yc2739@nyu.edu}{yc2739@nyu.edu}
\end{tabular}

\section{\faGraduationCap\ Education}
\datedsubsection{\textbf{New York University}, New York, America}{2016 -- Present}
\textit{Master student} in Electronics and Computer Engineering (ECE), expected May 2018
\datedsubsection{\textbf{Fudan University}, Shanghai, China}{2012 -- 2016}
\textit{B.S.} in Electronics Engineering (EE)

\section{\faUsers\ Experience}
\datedsubsection{\textbf{Shanghai Huajun Technology Ltd.} Shanghai, China}{2015 -- 2016}
\role{Internship}{Manager: Jinli Zhao}
Brief introduction: Work on development of machine learning systems
\begin{itemize}
  \item A prediction system that used to collect and analyze user data.
  \item Language:python
\end{itemize}

\datedsubsection{\textbf{A face recognition project}}{Nov. 2015 -- Dec.2015}
\role{course final project}{Individual Projects}
Brief introduction: A project of face recognition on FPGA platform
\begin{itemize}
  \item project for DSP experiment final(a course)
  \item finish a face recognition system with accuracy more than 75 percent
  \item language:C, Matlab
\end{itemize}

\datedsubsection{\textbf{Wireless and Sensor Control of Neonatal Monitoring system }}{May. 2016 -- June.2016}
\role{graduation project}{Individual Projects}
My undergraduate diploma project.Finish a wireless control system based on CC3200 of a neonatl monitoring system
\begin{itemize}
  \item Work on CC3200 platform(both MCU and Wifi chips)
  \item Part of a whole neonatal monitoring system ,including sensor control ,simple data processing and connection with wifi network 
  \item language: C 
\end{itemize}



% Reference Test
%\datedsubsection{\textbf{Paper Title\cite{zaharia2012resilient}}}{May. 2015}
%An xxx optimized for xxx\cite{verma2015large}
%\begin{itemize}
%  \item main contribution
%\end{itemize}

\section{\faCogs\ Skills}
\begin{itemize}[parsep=0.5ex]
  \item Programming Languages: C  Python Matlab
  \item Platform: Linux MSP430 Ardiuno
\end{itemize}

\section{\faHeartO\ Honors and Awards}
\datedline{\textit{\nth{2} Prize}, Award on ardiuno development competition }{July. 2014}


\section{\faInfo\ Miscellaneous}
\begin{itemize}[parsep=0.5ex]
  %\item Blog: http://your.blog.me
  %\item GitHub: https://github.com/username
  \item Languages: English - Fluent, Mandarin - Native speaker
  \item Leader of  Shanghai Volunteer Organization of Popularization Science 
\end{itemize}

%% Reference
%\newpage
%\bibliographystyle{IEEETran}
%\bibliography{mycite}
\end{document}


% version 9.15
% !TEX program = xelatex

\documentclass{resume}
%\usepackage{zh_CN-Adobefonts_external} % Simplified Chinese Support using external fonts (./fonts/zh_CN-Adobe/)
%\usepackage{zh_CN-Adobefonts_internal} % Simplified Chinese Support using system fonts
%Frenzy.ai Ltd. New York,NY Aug.2017 – Present




\begin{document}
\pagenumbering{gobble} % suppress displaying page number

\name{Yancheng Chen}

\section{Basic Information}

\begin{tabular}{rl}
    \textsc{Address:}   & 724 5th Ave, Brooklyn, New York,NY 11232 \\
    \textsc{Phone:}     & 917-856-9718\\
    \textsc{email:}     & yc2739@nyu.edu \href{mailto:yc2739@nyu.edu}{}\\
    \textsc{GitHub:}    & https://github.com/StandAloneComplex666
    \textsc{LinkedIn:}  & 
\end{tabular}

\section{\faGraduationCap\ Education}
\datedsubsection{\textbf{New York University}, New York, America}{2016 -- Present}
\textit{Master student} in Electronics and Computer Engineering (ECE), expected May 2018
\datedsubsection{\textbf{Fudan University}, Shanghai, China}{2012 -- 2016}
\textit{B.S.} in Electronics Engineering (EE)

\section{\faUsers\ Experience}
\datedsubsection{\textbf{Frenzy.ai.} New York, NY}{August.2017 -- Now}
\role{Internship}{Internship Manager: James Chang}
Brief introduction: Data Engineer whose works including building RSS/XML feed parser, web crawlers, natrual
language processing, computer vision and machine learning components.
\begin{itemize}
  \item Cleaning raw image from blogs and Instagrams.Build a R-CNN net for recognizing clothes from people and
background.Build a net to classify different kinds of clothes.
  \item Frame:Tensorflow
  \item Language:python
\end{itemize}

\datedsubsection{\textbf{A face recognition project}}{Nov. 2015 -- Dec.2015}
\role{course final project}{Individual Projects}
Brief introduction: A project of face recognition on FPGA platform
\begin{itemize}
  \item final projectfinish a face recognition system with accuracy more than 75 percent
  \item language:C, Matlab
\end{itemize}

\datedsubsection{\textbf{Wireless and Sensor Control of Neonatal Monitoring system }}{May. 2016 -- June.2016}
\role{graduation project}{Individual Projects}
My undergraduate diploma project.Finish a wireless control system based on CC3200 of a neonatl monitoring system
\begin{itemize}
  \item Work on CC3200 platform(both MCU and Wifi chips)
  \item Part of a whole neonatal monitoring system ,including sensor control ,simple data processing and connection with wifi network 
  \item language: C 
\end{itemize}

\datedsubsection{\textbf{Regularization methods of deep learning(CNN) }}{February.2017 -- May.2017}
\role{Graduate Student Research}{Small Group Projects}
Implement different regularization methods of deep learning to overcome over-fitting problems in deep learning.My research was mainly focusing on two method:dropout and adding uncertainty factors.I compared to other mainstream regularization methods as well.
\begin{itemize}
  \item Framework: Pytorch
  \item Build a convolutional neural network based on MINST database to test how each regularization method performed under different settings.
  \item language: Python  
\end{itemize}


% Reference Test
%\datedsubsection{\textbf{Paper Title\cite{zaharia2012resilient}}}{May. 2015}
%An xxx optimized for xxx\cite{verma2015large}
%\begin{itemize}
%  \item main contribution
%\end{itemize}

\section{\faCogs\ Skills}
\begin{itemize}[parsep=0.5ex]
  \item Programming Languages: C  Python Matlab Java
  \item Platform: Linux MSP430 Ardiuno
\end{itemize}

\section{\faHeartO\ Honors and Awards}
\datedline{\textit{\nth{2} Prize}, Award on Ardiuno development competition }{July. 2014}
\datedline{\textit{\nth{2} Prize}, Award on Electrical design competition of undergraduate students }{August. 2015}


\section{\faInfo\ Miscellaneous}
\begin{itemize}[parsep=0.5ex]
  %\item Blog: http://your.blog.me
  %\item GitHub: https://github.com/username
  \item Languages: English - Fluent, Mandarin - Native speaker
  \item Leader of  Shanghai Volunteer Organization of Popularization Science 
\end{itemize}

%% Reference
%\newpage
%\bibliographystyle{IEEETran}
%\bibliography{mycite}
\end{document}